\newcounter{softwaretnotecounter}

\iflanguage{ngerman}{
  \newcommand{\softwaresectiontitle}{Verwendete Software}
  \newcommand{\softwaretableimage}{Abbildung}
  \newcommand{\softwaretablename}{Name}
  \newcommand{\softwaretabledescription}{Beschreibung}
  \newcommand{\softwaretablecaptiontext}{Softwarekomponenten für das Projekt}

  \newcommand{\softwareltspicetext}{Simulationssoftware für elektronische Schaltungen, weitverbreitet zum Entwurf und Testen von Schaltungen.}
  \newcommand{\softwaremicrochipstudiotext}{Entwicklungsumgebung (IDE) zur Programmierung von Mikrocontrollern, insbesondere AVR und ARM.}
  \newcommand{\softwarefreecadtext}{Open-Source CAD-Software für 3D-Modellierung, geeignet für technische Konstruktionen und Design.}
  \newcommand{\softwarekicadtext}{Open-Source PCB-Design-Software, geeignet für die Erstellung von Leiterplattenlayouts.}
  \newcommand{\softwaredremeldigilabslicertext}{Software zum Vorbereiten und Slicen von 3D-Druck-Modellen, speziell für Dremel 3D-Drucker optimiert.}

  \newcommand{\softwarelogotext}{Die abgebildeten Logos sind markenrechtlich geschützte Symbole. Sie werden in dieser Dokumentation ausschließlich zur Identifikation der jeweiligen Softwareprodukte verwendet, ohne Werbeabsicht.}
}{
  \newcommand{\softwaresectiontitle}{Used Software}
  \newcommand{\softwaretableimage}{Figure}
  \newcommand{\softwaretablename}{Name}
  \newcommand{\softwaretabledescription}{Description}
  \newcommand{\softwaretablecaptiontext}{Software components for the project}

  \newcommand{\softwareltspicetext}{Simulation software for electronic circuits, widely used for designing and testing circuits.}
  \newcommand{\softwaremicrochipstudiotext}{Development environment (IDE) for programming microcontrollers, especially AVR and ARM.}
  \newcommand{\softwarefreecadtext}{Open-source CAD software for 3D modeling, suitable for technical constructions and design.}
  \newcommand{\softwarekicadtext}{Open-source PCB design software, suitable for creating printed circuit board layouts.}
  \newcommand{\softwaredremeldigilabslicertext}{Software for preparing and slicing 3D print models, specifically optimized for Dremel 3D printers.}

  \newcommand{\softwarelogotext}{The logos shown are protected by trademark law. They are used in this documentation solely for the identification of the respective software products, without any advertising intent.}
}

\section*{\softwaresectiontitle}
\begin{table}[!ht]
  \renewcommand{\arraystretch}{1.3}
  \centering
  \begin{threeparttable}
    \begin{tabularx}{\linewidth}{|M{2.4cm}|M{2.2cm}|X|}
      \hline
      \rowcolor{gray!20}
      \textbf{\softwaretableimage\tnote{*}} & \textbf{\softwaretablename} & \textbf{\softwaretabledescription} \\
      \hline
\ifsoftwarepagelineartechnology
      \stepcounter{softwaretnotecounter}
      \raisebox{-.25\height}{\includesvg[width=1.6cm,inkscapelatex=false]{./logo/linear_technology}} &
      LT-Spice\tnote{(\thesoftwaretnotecounter)} &
      \softwareltspicetext\\
      \hline
\fi
\ifsoftwarepagemicrochipstudio
      \stepcounter{softwaretnotecounter}
      \raisebox{-.25\height}{\includesvg[width=1.6cm,inkscapelatex=false]{./logo/microchip}} &
      Microchip Studio\tnote{(\thesoftwaretnotecounter)} &
      \softwaremicrochipstudiotext\\
      \hline
\fi
\ifsoftwarepagefreecad
      \stepcounter{softwaretnotecounter}
      \raisebox{-.25\height}{\includesvg[width=0.65cm,inkscapelatex=false]{./logo/freecad}} &
      FreeCAD\tnote{(\thesoftwaretnotecounter)} &
      \softwarefreecadtext\\
      \hline
\fi
\ifsoftwarepagekicad
      \stepcounter{softwaretnotecounter}
      \raisebox{-.25\height}{\includesvg[width=0.65cm,inkscapelatex=false]{./logo/kicad}} &
      KiCAD\tnote{(\thesoftwaretnotecounter)} &
      \softwarekicadtext\\
      \hline
\fi
\ifsoftwarepagedremeldigilabslicer
      \stepcounter{softwaretnotecounter}
      \raisebox{-.25\height}{\includesvg[width=1.6cm,inkscapelatex=false]{./logo/dremel}} &
      Dremel 3D DigiLab Slicer\tnote{(\thesoftwaretnotecounter)} &
      \softwaredremeldigilabslicertext\\
      \hline
\fi
    \end{tabularx}
    \caption{\softwaretablecaptiontext}
    \label{tab:software-programme}
    \begin{tablenotes}
      \footnotesize
      \item[*] \softwarelogotext
      \item[~]
\setcounter{softwaretnotecounter}{0}
\ifsoftwarepagelineartechnology
      \stepcounter{softwaretnotecounter}
      \item[(\thesoftwaretnotecounter)] \url{https://www.analog.com/en/resources/design-tools-and-calculators.html}
\fi
\ifsoftwarepagemicrochipstudio
      \stepcounter{softwaretnotecounter}
      \item[(\thesoftwaretnotecounter)] \url{https://www.microchip.com/en-us/tools-resources/develop/microchip-studio}
\fi
\ifsoftwarepagefreecad
      \stepcounter{softwaretnotecounter}
      \item[(\thesoftwaretnotecounter)] \url{https://www.freecad.org/downloads.php}
\fi
\ifsoftwarepagekicad
      \stepcounter{softwaretnotecounter}
      \item[(\thesoftwaretnotecounter)] \url{https://www.kicad.org/download/}
\fi
\ifsoftwarepagedremeldigilabslicer
      \stepcounter{softwaretnotecounter}
      \item[(\thesoftwaretnotecounter)] \url{https://www.dremel.com/at/de/digilab/3d-druckersoftware}  
\fi
    \end{tablenotes}
  \end{threeparttable}
\end{table}