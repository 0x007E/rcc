
\newcommand{\headerText}{\textbf{HTL-Rankweil}\\\texttt{Abteilung für Elektronik/Informatik}}
\newcommand{\titleLogoCaption}{Zusammengesetzter Cube}
\newcommand{\titleDescription}{\small Das RCC-Projekt basiert auf einer Platine mit einem ATtiny402 und zwei RGB-LEDs, die über den SPI-Bus gesteuert werden. Der Cube selbst wird über einen einzigen Taster gesteuert, der den Cube ein- und ausschaltet und zur Einrichtung der Farbe und Helligkeit der LEDs verwendet werden kann.\\}

\newcommand{\globalimportanttext}{Wichtig:}

\newcommand{\abstracttitle}{Einleitung}
\newcommand{\abstracttext}{
Das \texttt{RCC} (RGB LED Color Cube) Projekt ist ein eingebettetes System, das auf einer PCB mit einem \texttt{ATtiny402}\footnote{\url{https://link.sunriax.at/MecTz}} Mikrocontroller basiert, der zwei RGB-LEDs\footnote{\url{https://link.sunriax.at/nBUNH}} über SPI-Kommunikation steuert. Das System verfügt über eine Benutzeroberfläche mit einem einzigen Taster, um den Cube ein- und auszuschalten und die Farbe sowie die Helligkeitseinstellungen anzupassen. Es umfasst umfassende Hardware-Designs, Firmware und Bibliotheken für eine einfache Integration und Anpassung. Das Projekt nutzt KiCAD für das PCB-Design, FreeCAD für das Gehäuse und Microchip Studio für die Firmware-Entwicklung und legt dabei Wert auf Modularität und professionelle Produktionsbereitschaft durch automatisierte Build-Prozesse mit GitHub Actions. Dieses Setup bietet eine kompakte, vielseitige und programmierbare RGB-Beleuchtungslösung, die sich für Prototyping und Bildungszwecke in der Entwicklung von Elektronik- und eingebetteten Systemen eignet.
\newline
Die Montage der PCB erfolgt mit einer halbautomatischen Bestückungsmaschine, um eine präzise und effiziente Bauteilmontage sicherzustellen. Das Löten der PCB-Bauteile erfolgt mit einem Vapor-Phase-Lötverfahren, das eine gleichmäßige Wärmeverteilung und hochwertige Lötverbindungen gewährleistet. Der Batteriefach wird manuell mit herkömmlichen Lötmethoden verlötet, um eine sorgfältige Handhabung dieses spezifischen Bauteils zu ermöglichen. Das Gehäuse wird durch 3D-Druck hergestellt, was eine schnelle Prototypenerstellung mit hoher Genauigkeit und feinen Details ermöglicht. Darüber hinaus wird die Acrylplatte auf der Unterseite des Gehäuses mit einem CO\textsubscript{2} Laser Cutter zugeschnitten und graviert, um präzise Formen und individuelle Oberflächenmarkierungen zu gewährleisten.
\newline
Dieser Prozess garantiert ein professionell montiertes und fertiges Produkt, das automatisierte Fertigungstechniken mit manueller Präzision und fortschrittlichen Fertigungstechnologien kombiniert.
}

\newcommand{\pcbsectiontitle}{PCB Montage}
\newcommand{\pcbfigureassemblytopcaptiontext}{PCB Top-Bearbeitungs- und Montagelayout}
\newcommand{\pcbfigureassemblybottomcaptiontext}{PCB Bottom-Bearbeitungs- und Montagelayout}

\newcommand{\programmersectiontitle}{Firmware-Programmierung}
\newcommand{\programmerfigureprogrammercaptiontext}{Programmer}
\newcommand{\programmerintroductiontext}{Die Verwendung des \texttt{Programmiergeräts} zum Flashen des \texttt{ATtiny402} über die \texttt{UPDI}-Schnittstelle ist unkompliziert. Die erforderliche Software zum Flashen kann über das \texttt{RCC Programming Repository}\footnote{\url{https://github.com/0x007e/rcc_programmer}} heruntergeladen werden. Über die USB-to-UART-Brücke wird die Firmware auf den Mikrocontroller übertragen. Die Anwendung selbst bietet eine einfache Benutzeroberfläche, um die Firmware auf den \texttt{RCC} zu übertragen. Die Farbe und die Intensität der LEDs kann innerhalb der Anwendung angepasst werden, bevor sie auf den \texttt{RCC} übertragen werden.}
\newcommand{\programmertabledownloadlinkstitle}{RCC - Programmier-Software (Windows x64)}
\newcommand{\programmertabledownloadlinksclickoncetext}{ClickOnce-Installer, der automatisch aktualisiert wird, wenn neue Versionen verfügbar sind.}
\newcommand{\programmertabledownloadlinksportabletext}{Portable Version, die ohne Installation ausgeführt werden kann, aber manuell aktualisiert werden muss.}
\newcommand{\programmertabledownloadlinkscaptiontext}{Download-Links}
\newcommand{\programmerfigureinterfacecaptiontext}{Programmierschnittstelle}
\newcommand{\programmertablenoticetext}{Der \textbf{RCC}-Cube wird durch die USB-UART-Brücke (über die \texttt{Pogo}-Pins) mit \texttt{3V3} versorgt und benötigt keine Batterie. Es ist sicherzustellen, dass das \textbf{RCC}-Gerät \textbf{\textcolor{red}{nicht mit Batterie betrieben wird, um Schäden am Gerät zu vermeiden!}}}
\newcommand{\programmertablenoticecaption}{Programmiergerät-Hinweis}
\newcommand{\programmertableselfmadeupdititle}{Verwenden von externen UPDI-Adaptern}
\newcommand{\programmertableselfmadeupditext}{Ein selbstgebauter UPDI-Adapter kann auch innerhalb einer FT232 USB-to-UART-Brücke oder eines ähnlichen Moduls gemäß den Schaltplänen auf der \texttt{UPDI}\footnote{\url{https://github.com/0x007e/updi}}-Projektseite verwendet werden. Die Verbindungsanordnung finden Sie auf der Programmierer-Projektseite\footnote{\url{https://github.com/0x007e/rcc_programmer}} selbst.}
\newcommand{\programmertableselfmadeupdicaption}{UPDI-Adapter-Informationen}

\newcommand{\mechanicalsectiontitle}{Mechanische Montage}
\newcommand{\mechanicalintroductiontext}{Die mechanische Montage erfolgt durch das Zusammenfügen der 3D-gedruckten Gehäuseteile mit der lasergeschnittenen Acrylplatte. Die PCB wird mit den integrierten Montageschrauben im Gehäuse befestigt. Die Batterie muss montiert werden, bevor die PCB im Gehäuse platziert wird.}
\newcommand{\mechanicalsubsectionsteps}{Notwendige Schritte:}
\newcommand{\mechanicalfigurecaptionexplosion}{Explosionsansicht der mechanischen Montage}

\newcommand{\mechanicaltextfile}{./data/mechanicaltext.de.csv}
\newcommand{\mechanicalstepsfile}{./data/mechanicalsteps.de.csv}
