
\newcommand{\headerText}{\textbf{HTL-Rankweil}\\\texttt{Abteilung für Elektronik/Informatik}}
\newcommand{\titleLogoCaption}{Zusammengesetzter Cube}
\newcommand{\titleDescription}{\small Das RCC-Projekt basiert auf einer Platine mit einem ATtiny402 und zwei RGB-LEDs, die über den SPI-Bus gesteuert werden. Der Cube selbst wird über einen einzigen Taster gesteuert, der den Cube ein- und ausschaltet und zur Einrichtung der Farbe und Helligkeit der LEDs verwendet werden kann.\\}

\newcommand{\globalimportanttext}{Wichtig:}

\newcommand{\abstracttitle}{Einleitung}
\newcommand{\abstracttext}{
Das \texttt{RCC}-Projekt (RGB LED Color Cube) ist ein eingebettetes System, das einen \texttt{ATtiny402}\footnote{\url{https://link.sunriax.at/MecTz}} Mikrocontroller, welcher zwei RGB-LEDs\footnote{\url{https://link.sunriax.at/nBUNH}} über einen SPI-Bus steuert, verwendet. Das System verfügt über eine Benutzeroberfläche mit einem einzigen Taster, um den \texttt{RCC}-Cube ein- und auszuschalten und die Farbe sowie die Helligkeitseinstellungen anzupassen. Es umfasst ein einfaches Hardware-Design, sowie eine Firmware und Bibliotheken für eine einfache Integration und Anpassung. Das Projekt nutzt KiCAD für das PCB-Design, FreeCAD für das Gehäuse und Microchip Studio für die bereitstellung der Firmware und legt dabei Wert auf Modularität und professionelle Produktionsbereitschaft durch automatisierte Build-Prozesse mit GitHub Actions. Dieses Setup bietet eine kompakte, vielseitige und programmierbare RGB-Beleuchtungslösung, die sich für Prototyping und Bildungszwecke in der Entwicklung von Elektronik- sowie eingebetteten Systemen eignet.
\newline
Die Montage der Leiterplatte erfolgt mit einer halbautomatischen Bestückungsmaschine, um eine präzise und effiziente Bauteilmontage sicherzustellen. Das Löten der Bauteile erfolgt mit einer Dampfphasenlötanalge, die eine gleichmäßige Wärmeverteilung und hochwertige Lötverbindungen gewährleistet. Das Batteriefach wird manuell mit herkömmlichen Lötmethoden (Bleifrei) verlötet. Das Gehäuse wird mittels 3D-Druck hergestellt, was eine schnelle Prototypenerstellung, eine hohe Genauigkeit und feine Details ermöglicht. Darüber hinaus wird die Acrylplatte auf der Unterseite des Gehäuses mit einem CO\textsubscript{2}-Laser-Cutter zugeschnitten und graviert, um präzise Formen und individuelle Oberflächenmarkierungen herzustellen.
\newline
Dieser Prozessablauf garantiert ein professionell montiertes und fertiges Produkt, das automatisierte Fertigungstechniken mit manueller Präzision und fortschrittlichen Fertigungstechnologien kombiniert.
}

\newcommand{\pcbsectiontitle}{Leiterplattenmontage}
\newcommand{\pcbfigureassemblytopcaptiontext}{Leiterplatten Top-Bearbeitungs- und Montagelayout}
\newcommand{\pcbfigureassemblybottomcaptiontext}{Leiterpatten Bottom-Bearbeitungs- und Montagelayout}

\newcommand{\programmersectiontitle}{Firmware-Programmierung}
\newcommand{\programmerfigureprogrammercaptiontext}{Programmer}
\newcommand{\programmerintroductiontext}{Die Verwendung des \texttt{Programmiergeräts} zum Flashen des \texttt{ATtiny402} über die \texttt{UPDI}-Schnittstelle ist unkompliziert. Die erforderliche Software zum Flashen kann über das \texttt{RCC Programming Repository}\footnote{\url{https://github.com/0x007e/rcc_programmer}} heruntergeladen werden. Über die USB-to-UART-Brücke wird die Firmware auf den Mikrocontroller übertragen. Die Anwendung selbst bietet eine einfache Benutzeroberfläche, um die Firmware auf den \texttt{RCC} zu übertragen. Die Farbe und die Intensität der LEDs kann innerhalb der Anwendung angepasst werden.}
\newcommand{\programmertabledownloadlinkstitle}{RCC - Programmier-Software (Windows x64)}
\newcommand{\programmertabledownloadlinksclickoncetext}{ClickOnce-Installer, der automatisch aktualisiert wird, wenn neue Versionen verfügbar sind.}
\newcommand{\programmertabledownloadlinksportabletext}{Portable Version, die ohne Installation ausgeführt werden kann, aber manuell aktualisiert werden muss.}
\newcommand{\programmertabledownloadlinkscaptiontext}{Download-Links}
\newcommand{\programmerfigureinterfacecaptiontext}{Programmierschnittstelle}
\newcommand{\programmertablenoticetext}{Der \textbf{RCC}-Cube wird durch die USB-UART-Brücke (über die \texttt{Pogo}-Pins) mit \texttt{3V3} versorgt und benötigt keine Batterie. Es ist sicherzustellen, dass das \textbf{RCC}-Gerät \textbf{\textcolor{red}{nicht mit Batterie betrieben wird, um Schäden am Gerät zu vermeiden!}}}
\newcommand{\programmertablenoticecaption}{Programmiergerät-Hinweis}
\newcommand{\programmertableselfmadeupdititle}{Verwenden von externen UPDI-Adaptern}
\newcommand{\programmertableselfmadeupditext}{Ein selbstgebauter UPDI-Adapter kann auch innerhalb einer FT232 USB-to-UART-Brücke oder eines ähnlichen Moduls gemäß den Schaltplänen auf der \texttt{UPDI}\footnote{\url{https://github.com/0x007e/updi}}-Projektseite verwendet werden. Die Verbindungsanordnung finden Sie auf der Programmierer-Projektseite\footnote{\url{https://github.com/0x007e/rcc_programmer}} selbst.}
\newcommand{\programmertableselfmadeupdicaption}{UPDI-Adapter-Informationen}

\newcommand{\mechanicalsectiontitle}{Mechanische Montage}
\newcommand{\mechanicalintroductiontext}{Die mechanische Montage erfolgt durch das Zusammenfügen der 3D-gedruckten Gehäuseteile mit der lasergeschnittenen Acrylplatte. Die Leiterplatte wird mit den integrierten Montageschrauben im Gehäuse befestigt. Die Batterie muss montiert werden, bevor die Leiterplatte im Gehäuse platziert wird.}
\newcommand{\mechanicalsubsectionsteps}{Notwendige Schritte:}
\newcommand{\mechanicalfigurecaptionexplosion}{Explosionsansicht der mechanischen Montage}

\newcommand{\mechanicaltextfile}{./data/mechanicaltext.de.csv}
\newcommand{\mechanicalstepsfile}{./data/mechanicalsteps.de.csv}
